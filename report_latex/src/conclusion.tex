\section{Conclusion}
In this report, we provided a comprehensive overview of Out-Of-Distribution (OOD) detectors along with their timelines. 
We thoroughly discussed the main concepts behind each of the methods and formalized the scoring functions. 
Although there are numerous approaches for OOD detection in multi-class classifiers, the multi-label setting has not yet been fully explored. 
We have observed the implementation of existing methods in this domain and have also noted the emergence of new approaches such as YolOOD, 
which have produced interesting results as well. Some of the methods were implemented and quantitative results were produced. 
This could be used for benchmarking on new set of datasets. 
We attempted to convert GradNorm directly to the multi-label setting, but discovered that it does not produce a valid classifier. 

\section{Future Work}
There is a lot of potential areas to explore.
Existing approaches could be explored more and adapted for multi-label settings.
As suggested by~\cite{Zolfi2022}, using a different dataset that contains pixel-level annotations for training object detector will allow to learn accurate representations without irrelevant areas.
Another direction is to evaluate on large-scale dataset that mimic real-world settings.~\cite{hendrycksScalingOutofDistributionDetection2022,Huang2021}
